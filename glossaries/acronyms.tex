% to refer to an acronym, use:
%   \gls{<label>}   - This command prints the term associated with <label> passed as its argument. 
%                   If the hyperref package was loaded before glossaries it will also be hyperlinked 
%                   to the entry in glossary.
%   \glspl{<label>} - This command prints the plural of the defined term, other than that it 
%                   behaves in the same way as gls.
%   \Gls{<label>}   - This command prints the singular form of the term with the first character 
%                   converted to upper case.
%   \Glspl{<label>} - This command prints the plural form with first letter of the term converted 
%                   to upper case.
%   \glslink{<label>}{<alternate text>} - This command creates the link as usual, but typesets the 
%                   alternate text instead. It can also take several options which changes its 
%                   default behavior (see the documentation).
%   \glsfirst{<label>} - Prints as if it was the first time it is referred.

%--- institutions:
\newacronym {acronym_ist}{IST}{Instituto Superior Técnico}
\newacronym {acronym_isr}{ISR}{Institute for Systems and Robotics}

%--- generic terms:
\newacronym[plural={two dimensions}]
            {acronym_2d}{\mbox{2-D}}{two-dimensional}
\newacronym[plural={three dimensions}]
            {acronym_3d}{\mbox{3-D}}{three-dimensional}
\newacronym {acronym_cpu}{CPU}{central processing unit}
\newacronym {acronym_udp}{UDP}{user datagram protocol}
\newacronym {acronym_c_lang}{C}{C programming language}
\newacronym {acronym_cpp_lang}{C++}{C++ programming language}
\newacronym {acronym_rms}{RMS}{root mean square}
\newacronym {acronym_nis}{NIS}{normalized innovation square}

%--- sensor:
\newacronym {acronym_ladar}{LADAR}{laser detection and ranging system}
\newacronym {acronym_lidar}{LIDAR}{light detection and ranging system}
\newacronym {acronym_ins}{INS}{inertial navigation system}
\newacronym {acronym_gps}{GPS}{global positioning system}
\newacronym {acronym_cors}{CORS}{continuously operating reference station}
\newacronym {acronym_imu}{IMU}{inertial measurement unit}
\newacronym {acronym_ahrs}{AHRS}{attitude and heading reference system}

\newacronym {acronym_eci}{ECI}{earth-centered inertial}
\newacronym {acronym_ecef}{ECEF}{earth-centered earth-fixed}
\newacronym {acronym_ned}{NED}{north-east-down}
\newacronym {acronym_enu}{ENU}{east-north-up}


%--- platforms:
\newacronym {acronym_uav}{UAV}{unmanned aerial vehicle}
\newacronym [longplural={autonomous surface craft}]
            {acronym_asc}{ASC}{autonomous surface craft}
\newacronym {acronym_asv}{ASV}{autonomous surface vessel}

%--- methods and theories:
\newacronym {acronym_lqr}{LQR}{linear quadratic regulator}

\newacronym {acronym_mpc}{MPC}{model predictive control}
\newacronym {acronym_gpc}{GPC}{generalized predictive control}
\newacronym {acronym_nmpc}{NMPC}{nonlinear model predictive control}

\newacronym {acronym_lpv}{LPV}{linear parameter varying}
\newacronym[longplural={linear matrix inequalities}]
            {acronym_lmi}{LMI}{linear matrix inequality}
\newacronym[longplural={power spectral densities}]
            {acronym_psd}{PSD}{power spectral density}
\newacronym {acronym_sdg}{SDG}{statistical discrete gust}

\newacronym {acronym_icp}{ICP}{iterative closest point}
\newacronym {acronym_ml}{ML}{maximum likelihood}

\newacronym {acronym_ekf}{EKF}{extended Kalman filter}
\newacronym {acronym_slam}{SLAM}{simultaneous localization and mapping}
\newacronym {acronym_jcbb}{JCBB}{joint compatibility branch and bound}
\newacronym {acronym_ccda}{CCDA}{combined constrained data association}

\newacronym {acronym_gas}{GAS}{globally asymptotically stable}
\newacronym {acronym_ugas}{UGAS}{uniform global asymptotic stability}
\newacronym {acronym_ltv}{LTV}{linear time-varying}
\newacronym {acronym_uco}{UCO}{uniformly completely observable}

%\newacronym {acronym_}{}{}


