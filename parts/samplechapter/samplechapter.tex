\chapter[Sensor-based SLAM]{%
    \begin{minipage}[c]{0.78\textwidth}
        \begin{flushright} 
            Sensor-based Simultaneous \\
            Localization and Mapping
        \end{flushright} 
    \end{minipage}
%\parbox[t]{0.9\textwidth}{\flushright Sensor-based Simultaneous Localization and Mapping}
}
%Sensor-based Simultaneous Localization and Mapping}
\label{ch:slam}

This chapter presents the design, observability conditions, convergence analysis, 
and experimental validation of a sensor-based filter for \gls{acronym_slam}, 
with \gls{acronym_gas} error dynamics. 
This filter has straightforward application to \glspl{acronym_uav} in \gls{acronym_gps}-denied 
environments, using acceleration and angular rate inertial measurements, a laser 
scanning device, and an altitude sensor. 
The \gls{acronym_slam} problem is formulated in a sensor-based framework and modified 
in such a way that the underlying system structure can be regarded as \gls{acronym_ltv} 
for observability and filter design purposes, from which a linear Kalman filter with 
\gls{acronym_gas} error dynamics follows naturally. 
%% second part of the chapter 
Furthermore, an online earth-fixed trajectory and map estimation algorithm 
is developed using a computationally efficient and numerically robust methodology. 
This algorithm follows from the estimation solution provided by the sensor-based 
\gls{acronym_slam} filter, by formulating an optimization problem with a closed-form 
solution. 
The uncertainty description of this solution is also derived resorting to perturbation 
theory, both for the earth-fixed trajectory and map. 
%% results 
The performance and consistency validation of the proposed sensor-based \gls{acronym_slam} 
filter and the earth-fixed trajectory and map estimation method are successfully assessed 
with real data, acquired indoors using a quadrotor instrumented platform.

% Brief History of SLAM
Over the past decades the research community has devoted tremendous 
effort in the field of probabilistic \gls{acronym_slam}. 
The seminal works that established the statistical foundations for describing the 
relationships between landmarks and their correlations include \cite{smith:1986},
\cite{durrant:1988}, and \cite{smith:1990}. 
Further research showed that a 
full and consistent solution to this problem would require all the vehicle 
pose and map variables to be considered in a single, and potentially large, state vector. 
This requirement made the problem intrinsically nonlinear and one of the 
technical solutions emerging from this challenge was to use the much celebrated
\gls{acronym_ekf} \cite{csorba:1996,csorba:1997}. 
A detailed survey and tutorial on \gls{acronym_slam} can be found in \cite{durrant:2006}, 
\cite{bailey:2006}, \cite{thrun:2005}, and references therein, which include a significant 
number of successful implementations of \gls{acronym_slam} algorithms in real world scenarios.
The association of measured and state landmarks is one of 
the major sources of inconsistency in \gls{acronym_slam} algorithms and considerable effort 
has been put into this issue.
Several strategies have emerged, namely, the simplistic nearest neighbor (NN) 
validation gating, \gls{acronym_jcbb} \cite{neira:2001}, \gls{acronym_ccda} 
\cite{bailey:2002}, and other strategies, such as multi-hypothesis data 
association or particle filters, that imply changes or a different algorithm. 
%Another strategy to achieve better consistency and to reduce the computational
%complexity of the \gls{acronym_slam} algorithm is the use of submaps \cite{tardos:2002}.

An instrumented quadrotor was hand-driven along a path of about 60 meters in an 
indoor environment with a loop, as shown in Figure \ref{slam:fig:map}, at an 
average speed of 0.4 m/s.
The trajectory described by the vehicle starts near the middle 
%, at position $(-2.75,-3.67)$ m of the last body frame
and circulates counterclockwise until some of the first landmarks detected are once 
again visible, at the lower right corner.
This custom built quadrotor \gls{acronym_uav}, property of the \gls{acronym_isr}, is 
equipped with a MEMSENS nanoIMU, a Maxbotix XL sonar for altitude measurements, and a 
Hokuyo UTM-30LX laser scanning device that provides horizontal profiles of the surroundings.
\newcommand{\figNameA}[0]{20110729_run20111208-1135_cardume_part0161}
\begin{figure}
    \centering
    \includegraphics[width=14cm]{\figNameA_MapI_blueprint_MD}\\
    \caption[Map and trajectory in the earth-fixed reference frame.]
            {   Map and trajectory in the earth-fixed reference frame, obtained 
                using an optimization-based trajectory and map estimation
                algorithm, with the floor blueprint in the background.}
    \label{slam:fig:map}
\end{figure}


